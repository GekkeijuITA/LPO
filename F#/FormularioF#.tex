\documentclass[12pt]{article}
\usepackage{listings}
\usepackage[italian]{babel}
\usepackage{geometry}
\usepackage{xcolor}

\lstdefinelanguage{FSharp}%
  {morekeywords={let, in, rec, function, fun, type, member, match, with, try, begin, end, new, null, val, mutable, return, yield, do, use, if, then, else, elif, for, while, to, downto, match, with, open, module, namespace, struct, interface, inherit, abstract, override, as, of, when, failwith},%
   sensitive=true,%
   morecomment=[l][\color{gray}]{///},%
   morecomment=[l][\color{gray}]{//},%
   morecomment=[s][\color{gray}]{(*}{*)},%
   morestring=[b]",%
   morestring=[b]',%
   morestring=[b]""",%
   morestring=[b]'''%
  }[keywords,comments,strings]%

\lstset{ %
  language=FSharp, 
  basicstyle=\ttfamily\small, 
  keywordstyle=\bfseries\color{blue}, 
  stringstyle=\color{red}, 
  commentstyle=\color{gray}\itshape, 
  breaklines=true, 
  columns=fullflexible, 
  keepspaces=true, 
  showspaces=false, 
  showstringspaces=false
}

\geometry{margin=2cm}
\let\olditemize\itemize
\renewcommand\itemize{\olditemize\setlength\itemsep{0em}}



\title{Formulario F\#}
\author{}
\date{}

\begin{document}
\maketitle
\section{Funzioni}
\begin{lstlisting}[language=FSharp]
    let nomeFunzione arg1 arg2 = <corpo funzione>
    let rec nomeFunzioneRicorsiva arg1 arg2 = <corpo funzione>
    fun arg1 arg2 -> <corpo funzione> // funzione anonima
\end{lstlisting}
\section{Liste}
\begin{itemize}
    \item [1;2;3] $\rightarrow$ head::tail, head = 1, tail = [2;3]
    \item Usiamo @ per appendere una \textbf{lista} in coda ad un'altra, usiamo :: per concatenare un \textbf{elemento} in testa ad una lista
\end{itemize}
\section{Pattern Matching}
\begin{lstlisting}[language=FSharp]
    match x with
    | 1 -> "Uno"
    | 2 -> "Due"
    | _ -> "Altro"
\end{lstlisting}
Si può usare il \texttt{when} per aggiungere condizioni
\begin{lstlisting}[language=FSharp]
    match x with
    | x when x > 0 -> "Positivo"
    | x when x < 0 -> "Negativo"
    | _ -> "Zero"
\end{lstlisting}
\section{Eccezioni}
\subsection{Per lanciarle}
\begin{lstlisting}[language=FSharp]
    [altro codice] failwith "Messaggio di errore"
\end{lstlisting}
\subsection{Per gestirle}
\begin{lstlisting}[language=FSharp]
    try
        [codice che potrebbe lanciare eccezioni]
    with
        | e -> printfn "Errore: %s" e.Message
\end{lstlisting}
\section{Tipi}
\begin{lstlisting}[language=FSharp]
    type nomeTipo = | Costruttore1 [of tipo1 * tipo2 * ...] | ...
\end{lstlisting}
\section{Esempi di funzioni ricorsive e con accumulatore}
\subsection*{Accoppiamento elemento indice}
    \lstinputlisting[language=FSharp]{Soluzioni FS/Pairwise.fs}
\subsection*{Somma elemento per elemento}
    \lstinputlisting[language=FSharp]{Soluzioni FS/SumWise.fs}
\subsection*{Selezione elementi in base a condizione}
    \lstinputlisting[language=FSharp]{Soluzioni FS/Select.fs}
\subsection*{Unione di liste applicando funzione f}
    \lstinputlisting[language=FSharp]{Soluzioni FS/Fuse.fs}
\subsection*{Somma elementi lista (due elementi alla volta)}
    \lstinputlisting[language=FSharp]{Soluzioni FS/Agg.fs}
\subsection*{Duplicazione elementi lista}
    \lstinputlisting[language=FSharp]{Soluzioni FS/Dup.fs}
\end{document}