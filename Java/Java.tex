\documentclass[10pt]{article}
\usepackage[italian]{babel}
\usepackage{tikz, tikz-qtree}
\usepackage{amsmath}
\usepackage{amssymb}
\usepackage{listings}
\usepackage{xcolor}
\usepackage{geometry}

\definecolor{codegreen}{rgb}{0,0.6,0}
\definecolor{codegray}{rgb}{0.5,0.5,0.5}
\definecolor{codepurple}{rgb}{0.58,0,0.82}
\definecolor{backcolour}{rgb}{0.95,0.95,0.92}

\lstdefinestyle{mystyle}{  
    commentstyle=\color{codegreen},
    keywordstyle=\color{magenta},
    numberstyle=\tiny\color{codegray},
    stringstyle=\color{codepurple},
    basicstyle=\ttfamily\footnotesize,
    breakatwhitespace=false,         
    breaklines=true,                 
    captionpos=b,                    
    keepspaces=true,                   
    numbersep=5pt,                  
    showspaces=false,                
    showstringspaces=false,
    showtabs=false,                  
    tabsize=2
}
\lstset{style=mystyle}
\geometry{margin=2cm}

\title{Java}
\author{}
\date{}

\begin{document}
\maketitle
\begin{tikzpicture}
    \Tree [ .double [ .float [ .long [ .int [ .short [ byte ] ] char ] ] ] ]
\end{tikzpicture}
\begin{tikzpicture}
    \Tree [ .Object [ [ .Number Byte Short Integer Long Float Double ] Character Boolean Array String ] ]  
\end{tikzpicture}
\section{Regole}
$T_{1} \text{ e } T_{2}$ sono due tipi:
\begin{itemize}
    \item $T_{1}\leq T_{2}\Rightarrow T_{1}[]\leq T_{2}[] \leq Object$
    \item $T$ primitivo: $T[]\leq Object$ e l'unico array compatibile con $T[]$ è sè stesso
\end{itemize}
\textbf{Esempi:}
\begin{itemize}
    \item \texttt{String[] $\leq$ Object[] $\leq$ Object}
    \item \texttt{Integer[] $\leq$ Number[] $\leq$ Object}
    \item \texttt{Integer[] $\nleq$ Long[]}
    \item \texttt{int[] $\leq$ Object}
    \item \texttt{int[] $\nleq$ Integer[]}
    \item \texttt{int[] $\nleq$ Object[]}
    \item \texttt{int[] $\nleq$ long[]}
\end{itemize}
\section{Definizioni}
\begin{itemize}
    \item \textbf{Boxing}: è il processo di conversione di un tipo primitivo nella classe oggetto wrapper corrispondente (es. \texttt{int} in \texttt{Integer})
    \item \textbf{Unboxing}: è il processo contrario del boxing, ovvero la conversione di un oggetto wrapper in un tipo primitivo (es. \texttt{Integer} in \texttt{int})
\end{itemize}
\section{Esercizio dei Tipi (tips)}
\begin{itemize}
    \item Se la richiesta ha tipo dinamico (\texttt{P p2 = h}) allora guardiamo le funzioni della classe 'statica' (P) e le funzioni che ha in comune con la classe dinamica (H) verranno sovrascritte dalla classe dinamica.
    \item Ci potrebbero essere 2 tentativi a runtime quando gli passiamo un oggetto, il primo tentativo è quello che cerca la funzione con firma che possa contenere l'oggetto o dell'oggetto stesso, il secondo tentativo è quello che fa con l'unboxing. 
    \item Se la richiesta ha tipo H che estende P allora guardiamo le funzioni di H e quelle di P dove quelle di H fanno override di quelle di P.
\end{itemize}
\section{Tips}
\begin{lstlisting}[language=Java]
    public Costruttore(Tipo... arg) {
        // Codice
    }
\end{lstlisting}
Quando ci sono i puntini, possiamo passare un numero variabile di argomenti. \texttt{arg} è considerato come array.
\section{Esercizi di Visitatori Liste}
        \subsubsection{VisitorTest.java}
            \lstinputlisting[language=Java]{VisitorExam/List/VisitorTest.java}
        \subsubsection{ListCons.java}
            \lstinputlisting[language=Java]{VisitorExam/List/ListCons.java}
        \subsubsection{EmptyList.java}
            \lstinputlisting[language=Java]{VisitorExam/List/EmptyList.java}
        \subsubsection{Length.java}
            \lstinputlisting[language=Java]{VisitorExam/List/Length.java}
\section{Esercizi di Visitatori Classi}
        \subsubsection{VisitorTest.java}
            \lstinputlisting[language=Java]{VisitorExam/Class/VisitorTest.java}
        \subsubsection{ClassEntity.java}
            \lstinputlisting[language=Java]{VisitorExam/Class/ClassEntity.java}
        \subsubsection{InstanceEntity.java}
            \lstinputlisting[language=Java]{VisitorExam/Class/InstanceEntity.java}
        \subsubsection{SuperClassOf.java}
            \lstinputlisting[language=Java]{VisitorExam/Class/SuperClassOf.java}
\section{Esercizi di Iteratori}
        \subsection{Liste}
            \subsubsection{Test.java}
                \lstinputlisting[language=Java]{IteratorExam/Pow/Test.java}
            \subsubsection{PowIterator.java}
                \lstinputlisting[language=Java]{IteratorExam/Pow/PowIterator.java}

        \subsection{Classi}
            \subsubsection{IteratorTest.java}
                \lstinputlisting[language=Java]{IteratorExam/StringArray/IteratorTest.java}
            \subsubsection{StringArrayRevIterator.java}
                \lstinputlisting[language=Java]{IteratorExam/StringArray/StringArrayRevIterator.java}
\end{document}