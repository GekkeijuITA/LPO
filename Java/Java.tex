\documentclass[10pt]{article}
\usepackage[italian]{babel}
\usepackage{tikz, tikz-qtree}
\usepackage{amsmath}
\usepackage{amssymb}
\usepackage{listings}
\usepackage{xcolor}

\definecolor{codegreen}{rgb}{0,0.6,0}
\definecolor{codegray}{rgb}{0.5,0.5,0.5}
\definecolor{codepurple}{rgb}{0.58,0,0.82}
\definecolor{backcolour}{rgb}{0.95,0.95,0.92}

\lstdefinestyle{mystyle}{
    backgroundcolor=\color{backcolour},   
    commentstyle=\color{codegreen},
    keywordstyle=\color{magenta},
    numberstyle=\tiny\color{codegray},
    stringstyle=\color{codepurple},
    basicstyle=\ttfamily\footnotesize,
    breakatwhitespace=false,         
    breaklines=true,                 
    captionpos=b,                    
    keepspaces=true,                 
    numbers=left,                    
    numbersep=5pt,                  
    showspaces=false,                
    showstringspaces=false,
    showtabs=false,                  
    tabsize=2
}

\lstset{style=mystyle}

\title{Java}
\author{}
\date{}

\begin{document}
\maketitle
\begin{tikzpicture}
    \Tree [ .double [ .float [ .long [ .int [ .short [ byte ] ] char ] ] ] ]
\end{tikzpicture}
\begin{tikzpicture}
    \Tree [ .Object [ [ .Number Byte Short Integer Long Float Double ] Character Boolean ] ]  
\end{tikzpicture}
\section{Regole}
$T_{1} \text{ e } T_{2}$ sono due tipi:
\begin{itemize}
    \item $T_{1}\leq T_{2}\Rightarrow T_{1}[]\leq T_{2}[] \leq Object$
    \item $T$ primitivo: $T[]\leq Object$ e l'unico array compatibile con $T[]$ è sè stesso
\end{itemize}
\textbf{Esempi:}
\begin{itemize}
    \item \texttt{String[] $\leq$ Object[] $\leq$ Object}
    \item \texttt{Integer[] $\leq$ Number[] $\leq$ Object}
    \item \texttt{Integer[] $\nleq$ Long[]}
    \item \texttt{int[] $\leq$ Object}
    \item \texttt{int[] $\nleq$ Integer[]}
    \item \texttt{int[] $\nleq$ Object[]}
    \item \texttt{int[] $\nleq$ long[]}
\end{itemize}
\section{Esercizio dei Tipi (tips)}
\begin{itemize}
    \item Se la richiesta ha tipo dinamico (\texttt{P p2 = h}) allora facciamo "l'intersezione" delle funzioni di P e H e diamo la precedenza alle funzioni di H
    \item Ci potrebbero essere 2 tentativi a runtime quando gli passiamo un oggetto, il primo tentativo è quello che cerca la funzione con firma che possa contenere l'oggetto o dell'oggetto stesso, il secondo tentativo è quello che fa con l'unboxing.
\end{itemize}
\section{Tips}
\begin{lstlisting}[language=Java]
    public Costruttore(Tipo... arg) {
        // Codice
    }
\end{lstlisting}
Quando ci sono i puntini, possiamo passare un numero variabile di argomenti.
\end{document}