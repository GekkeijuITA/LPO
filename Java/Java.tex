\documentclass[10pt]{article}
\usepackage[italian]{babel}
\usepackage{tikz, tikz-qtree}
\usepackage{amsmath}
\usepackage{amssymb}

\title{Java}
\author{}
\date{}

\begin{document}
\maketitle
\begin{tikzpicture}
    \Tree [ .double [ .float [ .long [ .int [ .short [ byte ] ] char ] ] ] ]
\end{tikzpicture}
\begin{tikzpicture}
    \Tree [ .Object [ [ .Number Byte Short Integer Long Float Double ] Character Boolean ] ]  
\end{tikzpicture}
\section{Regole}
$T_{1} \text{ e } T_{2}$ sono due tipi:
\begin{itemize}
    \item $T_{1}\leq T_{2}\Rightarrow T_{1}[]\leq T_{2}[] \leq Object$
    \item $T$ primitivo: $T[]\leq Object$ e l'unico array compatibile con $T[]$ è sè stesso
\end{itemize}
\textbf{Esempi:}
\begin{itemize}
    \item \texttt{String[] $\leq$ Object[] $\leq$ Object}
    \item \texttt{Integer[] $\leq$ Number[] $\leq$ Object}
    \item \texttt{Integer[] $\nleq$ Long[]}
    \item \texttt{int[] $\leq$ Object}
    \item \texttt{int[] $\nleq$ Integer[]}
    \item \texttt{int[] $\nleq$ Object[]}
    \item \texttt{int[] $\nleq$ long[]}
\end{itemize}
\end{document}