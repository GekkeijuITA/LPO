\documentclass[12pt]{article}
\usepackage{listings}
\usepackage[italian]{babel}
\usepackage{geometry}
\usepackage{xcolor}
\usepackage{tikz,tikz-qtree}

\title{Grammatiche}
\author{}
\date{}

\geometry{margin=2cm}
\let\olditemize\itemize
\renewcommand\itemize{\olditemize\setlength\itemsep{0em}}

\begin{document}
\maketitle
\section{Associatività}
\begin{itemize}
    \item \textbf{Associatività sinistra}: la grammatica è ricorsiva a sinistra (es. $Exp::= Exp|Atom$)
    \item \textbf{Associatività destra}: la grammatica è ricorsiva a destra (es. $Exp::= Atom|Exp$)
\end{itemize}
\section{Ambiguità}
Una grammatica ambigua se per un'espressione $E$ esistono \textbf{almeno} due alberi diversi.
Come facciamo a ristrutturare la grammatica in modo da renderla non ambigua?\\
Un modo sarebbe quello di usare la notazione postfissa e prefissa, altrimenti si rende la grammatica ricorsiva a sinistra o a destra oppure si aggiungono dei livelli alla grammatica. Di solito quelli che danno problemi sono le operazioni che non sono ricorsive (nella grammatica) oppure i livelli inferiori che chiamano quelli superiori.
\section{Esempio da esame}
Exp ::= Exp + Atom $|$ Exp * Exp $|$ Atom\\
Atom ::= a $|$ b $|$ (Exp)\\
E' ambigua per l'espressione $a*b*a$.
\begin{center}
    \begin{tikzpicture}
        \Tree [.Exp [.Exp [.Exp [.Atom a ] ] * [.Exp [.Atom b ] ] ] * [.Exp [.Atom a ] ] ]
    \end{tikzpicture}
    \begin{tikzpicture}
        \Tree [.Exp [.Exp [.Atom a ] ] * [.Exp [.Exp [.Exp [.Atom b ] ] * [.Exp [.Atom a ] ] ] ] ]
    \end{tikzpicture}
\end{center}
Per renderla non ambigua basta modificare la produzione per  l'operatore * forzando, per esempio, l'associatività a sinistra:\\
Exp ::= Exp + Atom $|$ Exp * Atom $|$ Atom\\
Atom ::= a $|$ b $|$ (Exp)
\end{document}